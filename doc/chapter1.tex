\section{Topics of interest}
Here we discuss in a little more detail the previously outlined topics.

\subsection{Concurrency}
Ada is considered to have a good concurrency support. 

% TODO Revise this part 
\subsection{Sockets}
Sockets are a good topic on this case study, since sockets are OS dependent
therefore will provide some insight on how the language and application will
behave on different platforms. 

\subsection{Protocol Implementation}
A topic that I have came accross in programming languages that is somewhat
interesting to see in technical implementation, are protocols. Notably 
\textbf{Erlang} makes this a breeze. Investigating this in Ada will be 
interesting. 

For this small project, we'll be implementing 4 parts of the Http protocol. 
Namely the GET, POST, PUT and DELETE methods. We refer to the RFC2616 manual 
for the definition of behaviours of these methods. Once they are implemented in
this case study using Ada, the project shall be tested by using any standard
browser such as \textit{Firefox} or \textit{Chromium}.

For the reader we extract the definitions and behaviour of these four methods.
\subsubsection{Overview of Http Protocol}
We must add more stuff here.

\subsubsection{Http 1.1 / GET}
We must add more stuff here.

\subsubsection{Http 1.1 / POST}
We must add more stuff here.

\subsubsection{Http 1.1 / PUT}
We must add more stuff here.

\subsubsection{Http 1.1 / DELETE} 
We must add more stuff here.

\subsection{First Steps}
The first steps in an Ada application is to define the \textit{project manager 
file} for the given project as outlined on the manual \cite{GNATintro}.

%% TODO: Need to check if this is indeed the case. 
The GNAT project file abstracts a lot of the building details into this file.
It will take care even about other language builds, meaning that if you wish
to create, for example C extensions to the language, you could probably achieve
this easily, as opposed to tedious platform dependent makefiles. 

Another added advantage is that GNAT Project Files also take into consideration 
the file naming conventions. Another automated feature in this file is automatic
building of external libraries. There are various other ways to achieve this,
either on a makefile level, or alternatively using features in certain SCM tools
to clone and build the libraries automatically. However having everything in one
file allows for better, quick communication on the specification of ,
albeit coupling very specific tr

