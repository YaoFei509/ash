\section{Appendix}
The appendix will give some extra simplified information which might help understand the smaller parts of the given case study. Explanations and code listings are provided as best as possible.

\subsection{Networking}
A lot of the networking will seem very familiar to you if you have used BSD C Sockets. Provided are two Ada procedures. One listener, and one sender. The Listener listens for incoming connections. If a message is intercepted, the message is uppercased, and sent back to the client. 
The client simply sends a string message to the listener. The listener listens at port 3000 (it is assumed these procedures run on your local machine, together).

\lstinputlisting[language=Ada,caption=Simple Listener]{./src/appendix/simple_listener.adb}
\lstinputlisting[language=Ada,caption=Simple Sender]{./src/appendix/simple_sender.adb}

\subsection{Task Oriented Concurrency Model}
Concurrency in Ada uses tasks. Here is a small example to demonstrate this feature of the language that exploits a very interesting way of achieving concurrent tasks. If more detail is required, refer to \cite{AdaTasking}

\subsubsection{Simple Concurrent Program}
This is the simplest example in using tasks in Ada. The example was adapted from notes found in \cite{AdaConcurrent}, and was also the first concurrent program I wrote for testing and understanding tasks, the first time around.
\lstinputlisting[caption=Task Types in Ada,language=Ada]{./src/appendix/task_types.adb}

This procedure would have the following output:
\lstinputlisting[caption=Output for the Task Types in Ada,language=]{./src/appendix/task_types_output.txt}

\section{Random Notes}
Some other miscellanious random notes, that might help in the future. 

\subsection{ADS / ADB file location}
If the \textit{gcc-ada} compiler is being used, the specification and body files can typically be found in the following directory
$$ /usr/lib/gcc/\$MACHTYPE/\$GCC-VERSION/adainclude/ $$

\subsection{Sanity Checking}
I wanted to test out if things were working properly every now and then. Here are a few commands that I used in order to do this.

\subsubsection{Using Telnet to conenct to a Http Server}
You can use telnet to connect to a http server, and write the headers manually, and receive the actual response. This helps at understanding a little more how the protocol works, and was used in order to debug the server.
\\
\lstinputlisting[language=,caption=Telnet to an Http Server]{./src/appendix/telnet-http.txt}

\subsubsection{Using ncat to dump packets to a binary file}
On the various stages of debugging, \textit{ncat} was used in order to dump the packets to a file, in order for inspection. This helped when I was facing trouble when the socket would send random garbage due to mistakes to the http servers, and have undesired consequences. 

\lstinputlisting[language=bash,caption=Using ncat to dump packets]{./src/appendix/ncat-command.txt}

Here is an investigation example of data sent through sockets from a test script. This is a GET request.
\lstinputlisting[language=bash,caption=Using ncat\,  a test script\, and xxd to hex dump]{./src/appendix/xxd-ncat-log.txt}
You can notice the CRLFs at the end of each request (that is the two bytes on the left, 0x0A, and 0x0D respectively).

And here is a small, priceless one-liner that has helped me debug some of my code during this case study. It uses ncat, and pipes the output to xxd, which in turn makes a hex dump.
\lstinputlisting[language=bash,caption=One liner to hex dump dynamically]{./src/appendix/dynamic-hexdump.txt}

\subsection{Text Manipulation}
We list some text manipulation example code in this section, if better understanding is required. First it would be better to familiarize oneself with the \textit{String} type; one can refer to \cite{AdaStringType}.

\subsubsection{Splitting}
A common task one wants to do with strings is split them. Refer to this \cite{AdaStringSplit}, to read upon string splitting. Notice that this was added on the GNAT platform.

\subsubsection{Regular Expressions}
Regular expressions don't quite exist in the Ada programming language. However the GNAT platform provides some libraries with similar effects. A simple example can be found at \cite{AdaRegularExpressions}.



